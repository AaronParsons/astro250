\documentclass[11pt]{article}
\usepackage{fullpage}
\usepackage{graphicx}
\usepackage{natbib}
\usepackage{hyperref}

\begin{document}
\title{Radio Skillz: Digital Lab 3}

\maketitle

\section*{Prerequisites}

\begin{itemize}
\item FIR Filters
\item Fast Fourier Transform
\item Network Programming
\end{itemize}

\section*{Materials}

\begin{itemize}
\item ROACH board
\item quad-ADC board
\item 200 MHz clock source
\item tone generator with AM/FM modulation
\end{itemize}

\section*{Some Thoughts}

\subsection*{Simulink}

\subsubsection*{Setting up a VNC Session}

From any *nix computer:
\begin{verbatim}
ssh -Y gbower@otto.eecs.berkeley.edu
$ ssh username@host.computer.edu
    (enter password)
$ vncserver
    (write down XX of "desktop is host.computer.edu:XX)
$ exit
\end{verbatim}
This will set up a VNC session for you to log into in the future.  Remember
your session number (XX).

When you are done with your session, it would be neighborly of you to free up the computer resources associated with your vnc session:
\begin{verbatim}
$ ssh username@host.computer.edu
$ vncserver -kill :XX
    (where XX is your VNC session from above)
$ exit
\end{verbatim}
Thanks!

\subsubsection*{Accessing Your VNC Session}

From any *nix computer:
\begin{verbatim}
$ vncviewer -via username@host.computer.edu localhost:XX
    (ssh password)
    (vnc password)
\end{verbatim}

All future commands are to be typed inside your VNC session.

%\subsection{About Simulink on Linux}
%
%The machine you are logged into is running a Linux desktop.
%It behaves somewhat differently than other desktop environments you may be
%used to.  Because Simulink sometimes opens windows that are larger than
%the size of the desktop, it may be useful to know that you can move these
%windows around by right-clicking, selecting "Move", and then dragging the window around.

\subsubsection*{Starting Matlab/Simulink}

To start Matlab/Simulink, type:

\begin{verbatim}
$ cd ~/mlib_devel
$ ./simmy_hitz
\end{verbatim}

This program is a script that sets up the licenses for all the tools we'll be
using, and then executes the program ``matlab''.  All further commands will be in
the matlab window, unless we say otherwise.

\section{Transmitting Data to a Host Computer via UDP}

We'll be starting with the BOF file from last week: the down-converter.

\begin{itemize}
\item starting from the automated data collection script that you wrote last week, add code (call it
``fir\_filter\_udp\_tx.py'') that continuously
reads new data from the BRAM and transmits it via UDP packets to a host computer connected to the ROACH.
\begin{itemize}
\item if it isn't already, I strongly recommend that you make this code object-oriented.
\item you'll want to use the python module socket.  Some useful snippets:
\begin{verbatim}
>>> import socket
>>> sock = socket.socket(socket.AF_INET, socket.SOCK_STREAM)
>>> sock.connect((host,port))
>>> sock.send(data)
...
>>> sock.recv(n_bytes)
\end{verbatim}
\item keep in mind that the ddc\_bram is continuously being written to, and ddc\_addr tells you where writing
is happening.  Only transmit new data, so that you get a continous stream of output.

\end{itemize}
\item write code for the host computer (call it ``fir\_filter\_udp\_rx.py'') that unpacks the values and 
plots them.  Some useful snippets:
\begin{verbatim}
>>> pylab.ion()
>>> line, = pylab.plot(...)
>>> line.set_ydata(...)
\end{verbatim}
\end{itemize}

\section{Optional Fun: Output to Audio}
\begin{itemize}
\item modify your UDP receiver code to massage the UDP packets and send them to a speaker.  Some useful snippets:
\begin{verbatim}
>>> import pyaudio
>>> p = pyaudio.PyAudio()
>>> fmt = p.get_format_from_wdith(bytes_per_samp)
>>> stream = p.open(format=fmt, channels=1, rate=44100, output=True)
>>> stream.write(data)
\end{verbatim}
Note that 44100 Hz, the typical rate for audio recordings, may not apply here.
\item we are writing real/imag data into the BRAM.  You'll either need to up-convert, or choose your LO
frequency carefully in order to be able to convert a complex stream to a real signal you can output to a speaker.
\end{itemize}

\section{Program an FIR Filter for FM to AM Conversion}

\begin{itemize}
\item begin with the file \url{~/aparsons/firfilter.mdl}
\item after the summing filter, a 4-tap (complex) FIR filter has been added with coefficients that 
are programmable from software registers.  You'll need to pick coefficients, and read the design to understand
what they mean.  Some keep in mind that the FIR filter operates on
the output of the accumulator that implements the summing
filter, which only outputs valid data every 1024$^{\rm th}$ sample.
\item first, program the FIR filter for unity gain to reproduce the output of the DDC.  Verify that this works.
\item next, choose a set of coefficients (and predict the filter response) to linearly convert frequency to
amplitude, ranging from 0 at negative frequencies to 1 at positive frequencies.
\item compare the predicted and measured response of the entire summing-filter/FIR-filter system.  
\item write a program ``fir\_filter\_response.py'' that illustrates all of these predictions, and, if you have
time, your measurements.
\item if you have the time, you could combine this output with the audio script to listen to FM radio
streaming over the internet from an FPGA-based tuner.  Cool.
\end{itemize}

\end{document}
