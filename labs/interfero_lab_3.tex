\documentclass[11pt]{article}
\usepackage{fullpage}
\usepackage{graphicx}
\usepackage{natbib}
\usepackage{hyperref}

\begin{document}
\title{Radio Skillz: Interferometry Lab 3}

\maketitle

\section*{Prerequisites}

\begin{itemize}
\item Self Calibration
\item Phase Calibration
\item Flux Calibration
\item Gridding
\item W Projection
\end{itemize}

\section*{Materials}

\begin{itemize}
\item laptop computer
\item Astronomical Interferometry in PYthon (AIPY)
\item data
\end{itemize}

\section*{Stuff}

Last time, you constructed a uv grid for synthesis imaging, using simplistic (round to nearest) gridding.
Today we're going to up our gridding game.  As before, the challenge is
that I hand you a UV dataset, and you make the picture.  You'll need AIPY and your previous code.
The data I'm giving you is still in the MIRIAD UV format, and as before,
the ``uvw coordinates'' returned by the UV file are not in wavelengths,
but nanoseconds.

\section{Add a Gridding Function to Your UV Grid}

\begin{itemize}
\item Currently, to grid, you are rounding to the nearest pixel.  You should change your code (by
sub-classing your UV Grid) to divide each visibility sample among some number of pixels (the ``footprint''
of you gridder).  I suggest the nearest 4 or 9.  Allow for an arbitrary function to be passed in
that is evaluated based on the actual uv coordinates of your sample to choose the weights for gridding
your sample in the UV plane.  Don't forget to put those same weights in your sampling grid, which
you use to determine your dirty beam.
\item Select a gridding function and image the data set from last time.  Compare the result with
your nearest-pixel gridding.  Predict numerically the effect the gridding has on your resultant image.
\end{itemize}

\section{Add W-Projection}

\begin{itemize}
\item Last time, you re-phased data that were observed with no $w$ component to point toward a new 
phase center, effectively removing the projected $w$ component
of each baseline relative to a region near phase center.  Now, you are going to be handed a dataset where
visibilities were fundamentally observed with different (and incompatible) $w$ components, such that
phasing is absolutely necessary.  As a first step, phase this data toward zenith and use standard gridding.
This is very wide-field data, so set the resolution of your UV grid to image the whole sky.
\item Inspect the resultant image, identifying sources with increasing distance from zenith.  Do you
see the widefield effects of the $w$ component?
\item Now incorporate $w$ projection.  This can be rather difficult to do efficiently, so we'll do it inefficiently instead.  Instead of putting $w$ projection in the analytic gridding you use above, generate an image-domain
picture of the phase errors introduced by the $w$ component.  Conjugate this (to ``undo'' the error), and 
then take the Fourier transform.  This is,
numerically, the $w$ kernel that should be used to grid your visibilities.  It depends on the $w$ of each baseline.
Convolve each (separately) gridded visibility by the appropriate $w$ component.  Don't forget to put all of this
in the sample grid as well.
\item Now image using $w$ projection.  Hopefully, you'll see a huge difference, both in how far out point
sources can be identified, and in a firmly defined horizon line.
\end{itemize}

\end{document}
