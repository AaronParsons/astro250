\documentclass[11pt]{article}
\usepackage{fullpage}
\usepackage{graphicx}
\usepackage{natbib}
\usepackage{hyperref}

\begin{document}
\title{Radio Skillz: Interferometry Lab 1}

\maketitle

\section*{Prerequisites}

\begin{itemize}
\item Correlators
\item Basic Interferometry
\item Delay Imaging
\end{itemize}

\section*{Materials}

\begin{itemize}
\item laptop computers with microphones
\item broad-band sound generator
\end{itemize}

\section{Build an Audio F-Engine}

We are going to build a software FX correlator for audio inputs.  The first step is to build the
F Engine which acquires the data from your computer and transmits it to a host that will perform
the cross-multiplication.

\begin{itemize}
\item write some code (call it ``audio\_f\_plot.py'') that acquires data from your microphone and plots the
spectrum in real time.  A useful snippet:
\begin{verbatim}
>>> import pyaudio as pa, numpy as n
>>> p = pa.PyAudio()
>>> stream = p.open(format=pa.paInt16,channels=1,rate=44100,
...         input=True,frames_per_buffer=1024)
>>> d = stream.read(NSAMPLES)
>>> d = n.frombuffer(d, dtype=n.int16)
\end{verbatim}
\item modify your plotting code (in a new file called ``audio\_f\_engine.py'') to transmit UDP packets
of spectrum data to a chosen port/IP address instead of plotting it.  You are going to have to define a 
UDP packet format
(agree on one with the whole class) that uniquely identifies streams coming from different computers, and
also includes the time that the spectrum was taken, so that data from different computers can be aligned in
time.  If you can create a time-stamp that is exactly equal for two packets that should be cross-multiplied
(even if there is slight skew in exactly when each computer sampled the data), it will make your life easier
later.  A useful snippet (repeated from digital lab 3):
\begin{verbatim}
>>> import socket
>>> sock = socket.socket(socket.AF_INET, socket.SOCK_STREAM)
>>> sock.connect((host,port))
>>> sock.send(data)
...
>>> sock.recv(n_bytes)
\end{verbatim}
\item modify your plotting code (in place as ``audio\_f\_plot.py'') to instead accept data over a UDP port
in the format you just defined, and plot it in real time.  Make sure this all still works.
\end{itemize}

\section{Build an Audio X Engine}

\begin{itemize}
\item write code (in a new file called ``audio\_x\_engine.py'') that time-aligns and cross-multiplies 
the spectral data arriving from two different audio F engines that are both transmitting to the (one) port that
you are listening to.  Create a real-time plot that shows both the auto- and cross-correlation spectra 
(i.e. the visibility).  For cross-correlations, there is important information in both the amplitudes and
phases of the visibilities.
\item use separate threads for receiving UDP data and plotting it.  A useful snippet:
\begin{verbatim}
>>> import threading
>>> class YourThread(threading.Thread):
...     def run(self):
            # do something
>>> t = YourThread()
>>> t.start()
\end{verbatim}
\end{itemize}

\section{Make a Delay Imager}

\begin{itemize}
\item modify your X engine code (in place) to also take the delay transform of your cross-correlation, producing
a delay ``image'' in real time.
\item think about your units, so that you can plot your delay image in physical (spatial/time) units.
\item If you have time, see if you can get your F engines to get better time synchronization, which will
make your delay plot more physically representative.  What do you think is the limiting factor in synchronization?
\end{itemize}

\end{document}
