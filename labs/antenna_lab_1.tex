\documentclass[11pt]{article}
\usepackage{fullpage}
\usepackage{graphicx}
\usepackage{natbib}
\usepackage{hyperref}

\begin{document}
\title{Radio Skillz: Antenna Lab 1}

\maketitle

\section*{Prerequisites}

\begin{itemize}
\item Reciprocity Theorem
\item Dipole Antennas
\item Impedance of Free Space
\item Radiometer Equation Applied to Telescopes
\end{itemize}

\section*{Materials}

\begin{itemize}
\item wood, wire, rope, saw, hammer, and other implements of destruction
\item function generator
\item oscilloscope
\item commercial FM antenna
\item cable of known impedance (say, 50$\Omega$)
\item transformer
\end{itemize}

%\section*{Some Thoughts}

\section{Building an Antenna}
\begin{itemize}
\item construct dipole arms out of wire, tuned to 137 MHz.  We'll be using these in future 
labs, so make them robust!
\item construct a structure for holding the dipoles in the desired configuration (including height above the
ground).
\item attach dipole arms to the structure
\item connectorize your dipole so you can drive it (yes, we are talking transmission) differentially with 
a signal over the cable of known impedance.
One way is to assign ground to one of the dipole arms, which makes it easy to construct for the purpose of this
lab.  However, since ground is tied (at some level) to the earth, which acts as a reflector in your dipole
antenna, this isn't ideal: from the earth perspective, only one arm of your dipole will be transmitting.  You'll
have a monopole.  Better is to turn a single-ended signal from your source into a differential signal via
a transformer, or as it is sometimes called, a ``balun''.
\end{itemize}

\section{Measuring the Impedance of Several Antennas}

\begin{itemize}
\item first, properly terminate the end of the cable you wish to use to drive a signal into your dipole, drive
the cable at one end with your function generator, and
then observe the signal amplitude (at the same end, as when we were measuring the impedance of a transmission line)
as a function of frequency via a scope.  Hopefully, the signal amplitude should 
be relatively constant.
\item next, connect the cable to your dipole, and then measure as a function of frequency the signal amplitude.
From this, you should be able to characterize $\Gamma$ for the cable-dipole interface, and thereby determine
versus frequency, the impedance of your dipole.  Graph this.  By the reciprocity theorem, this graph should tell
you how sensitive your dipole will be as a function of frequency.  Sweet.
\item Measure the VSWR of your antenna from 100-200 MHz.
\end{itemize}

\section{Calculate Signal Gain for Satellite Observations}
\begin{itemize}
\item suppose Orbcomm satellites in Low Earth Orbit (LEO) transmit 3W in a 50 kHz band centered at 137 MHz.  Let's
also assume that the sky is at a constant temperature of 300K, as are all receiver electronics.  For a 100 MHz
input bandwidth, how much 
amplification will you need in order to produce a signal amplitude that can be effectively sampled (but not
clipped) by
a CASPER Quad-ADC.  These ADC's are 8-bit and have an input voltage range of $\pm$1 V, with 50$\Omega$ termination. 
Assume that the Orbcomm satellites' have a beam about 30$^\circ$ across.
\item in the Orbcomm transmission band, what are the relative amplitudes of the signal and noise?  How long
will you have to integrate to get a 10-$\sigma$ detection?
\end{itemize}

\end{document}
