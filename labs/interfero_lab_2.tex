\documentclass[11pt]{article}
\usepackage{fullpage}
\usepackage{graphicx}
\usepackage{natbib}
\usepackage{hyperref}

\begin{document}
\title{Radio Skillz: Interferometry Lab 2}

\maketitle

\section*{Prerequisites}

\begin{itemize}
\item Measurement Equation
\item Synthesis Imaging
\item Earth Rotation Synthesis
\item Deconvolution
\item Radiometer Equation Applied to Interferometers
\end{itemize}

\section*{Materials}

\begin{itemize}
\item laptop computer
\item Astronomical Interferometry in PYthon (AIPY)
\item data
\end{itemize}

\section*{Stuff}

Today we're going to do synthesis imaging from (more or less) scratch.  The challenge is
that I hand you a UV dataset, and you make the picture.  Here's what you need.

\subsection*{AIPY}

AIPY is a package I wrote to do synthesis imaging in Python.  You need to install it.  Instructions are
at \url{https://casper.berkeley.edu/astrobaki/index.php/AIPY}.  I suggest checking out the GIT AIPY repo.

\subsection*{MIRIAD UV Files}

The data I'm giving you is in the MIRIAD UV format.  AIPY has wrappers (aipy.miriad) for interacting with
these files.  For a reference, check out \S3.2 in \url{http://setiathome.berkeley.edu/~aparsons/aipy/aipy.pdf}.
Basically, you open a file, and then read through the data sequentially.  Each time you read a spectrum,
you are told which antenna pair it belongs to, what time it was observed at, and the (hopefully, but
not always correct) $uv$ coordinates of the baseline:
\begin{verbatim}
>>> import aipy
>>> uv = aipy.miriad.UV('somefile.uv')
>>> print uv['nchan'] # number of frequency channels in each spectrum
>>> print uv['sfreq'] # starting frequency of each spectrum in GHz
>>> print uv['sdf'] # width of each channel in GHz
>>> for (uvw,t,bl),data,flags in uv.all(raw=True):
...     print uvw, t, bl, uv['pol'] # time is a Julian date
\end{verbatim}

It should be noted that the ``uvw coordinates'' returned by the UV file are not in wavelengths,
but rather, in nanoseconds.  Nanoseconds is a weird unit, but it has the nice property that,
when multiplied by a frequency in GHz, it gives you wavelengths.


\section{Construct a UV Grid}

\begin{itemize}
\item Construct a 2D matrix into which you will put the uv samples from the miriad file.  You'll have to
decide on the breadth and resolution of the matrix, and the decisions you make will have reprocussions on
the width and resolution of your image.  You may need to have a look through your UV file to see what
the maximum baseline length is.
\item Just looking forward a bit, you'll want to construct a class for your UV plane that includes
various methods for putting uv samples in, and keeping track of both your UV plane and your sampling weights.
\end{itemize}

\section{Grid the Data and Make and Image}

\begin{itemize}
\item Add in the samples from the UV plane from the Miriad data file.  You might want to look forward and
create some capability for rounding (or convolving) your samples down to the grid indices.  Remember to
add in the Hermitian conjugate data points.
\item Invert the UV matrix and the sampling matrix to make images of your sky and you dirty beam.
\item What are the coordinates of your image pixels?
\end{itemize}

\section{Re-Point}

\begin{itemize}
\item The data set you were handed was very friendly, in that it was pre-phased (with correctly
calculated $uv$ coordinates) to zenith.  Write some code that changes the phase center and generates
new $uv$ coordinates relative to that position.  (You can calculate the new $uv$ coordinates relative
to the old ones).
\item Apply the phasing that re-points the data in the file to the new phase center.
\item Grid the data at the new $uv$ coordinates
\item verify that the new image is a translation of the old image.
\end{itemize}

\end{document}
