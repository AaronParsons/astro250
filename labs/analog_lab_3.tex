\documentclass[11pt]{article}
\usepackage{fullpage}
\usepackage{graphicx}
\usepackage{natbib}
\usepackage{hyperref}

\begin{document}
\title{Radio Skillz: Analog Lab 3}

\maketitle

\section*{Prerequisites}

\begin{itemize}
\item Central Limit Theorem
\item Johnson-Nyquist Noise
\item Receiver Temperature
\item Radiometer Equation
\end{itemize}

\section*{Materials}

\begin{itemize}
\item 2W 50$\Omega$ resistor
\item power supply
\item oscilloscope w/ probes
\item minicircuits amplifiers
\item voltimeter
\item infrared (remote) thermometer
\end{itemize}

\section*{Some Thoughts}

\subsection*{Minicircuits Amplifiers}

Now that you've built your own amplifier, you have a feel for what goes on inside an amplifier.  We're
now going to use pre-packaged amplifiers that are commercially available (Minicircuits is a typical vendor).
These will be SMA in, SMA out, with two leads for $V_{cc}$ (+15V, for the parts we're using) and ground.
These amplifiers specify their gain in dB, and you do have to be a bit careful about signal levels.
As a general rule, the output of amplifiers should be connected (to a 50$\Omega$ load) before they are
powered on, so that they don't have to dissipate their output power internally.

\subsection*{Infrared Thermometer}

If you assume that objects radiate heat as a blackbodies (which we'll discuss in later labs), you
can measure their temperature remotely just by their infrared radiation.  The infrared thermometer
we will use in this lab is an infrared telescope (with a well-defined beam size) that is
calibrated to output a voltage in mV that is the temperature in Fahrenheit.  Use it wisely.

\subsection*{Interference}

In this lab, we're going to attempt to make a relatively sensitive measurement of the Johnson
noise on a 50$\Omega$ resistor.  It turns out that lots of things can produce signals at this 
level that can interfere with your measurement.  For one, if you make a loop that electrons can
flow around (which, after all, is what a circuit is), then a changing magnetic flux through that
loop produces an EMF that drives a current that sets up a voltage across the resistor.  This may
be the very first radio antenna we build in this class.  Unfortunately, that's not what we were
going for, so we need ways to avoid this.  Since the area inside the loop matters, you can try to
make your circuits physically narrow.  You might also try making a Faraday cage.

\subsection*{Writing Code for Lab}

This is the first lab where a significant portion of the lab is writing code.  We are going to keep
everyone's code in a class Git repository.  So the first thing you should do set up a github
account and fork the class repo: \url{https://github.com/AaronParsons/astro250.git} .
Next, clone your repo onto your computer.  (If you haven't yet, you should get Git, Python, Numpy,
and Pylab installed on your computer).  Inside astro250, add a directory with your name, and create
two files: central\_limit.py and radiometer\_test.py.  Commit, and push your changes back up to
your github account.  I'll pull your fork of astro250 into
my repo, along with everyone else's work.  From time to time, you should pull from my repo (you
can add a shorthand for it to your working copy on your computer with {\it git remote add ...})
so you can see what everyone else is doing, too.

\section{Calculating the Noise Figure of a Reciever}

\subsection*{Determine Amount of Gain Needed}
\begin{itemize}
\item Calculate how much amplification is needed to bring the Johnson noise on a $50\Omega$ resistor
up to at least the $\sim$1mV RMS level that you can measure on a scope.  You'll have to pick an appropriate
bandwidth, and you might want to check that you can get a filter that supplies that bandwidth. 
\item Looking at the data sheets for the minicircuits amplifiers, choose a number of gain stages.
\item Connect the gain stages (and the filter at the end), and power them at +15V.  IMPORTANT: Most amplifiers
want to have the output connected first, so that the output power has some place to go.
\end{itemize}


\subsection*{Estimate and Measure the Resistor Temperature}
\begin{equation}
P=A\epsilon \sigma T^4
\end{equation}
\begin{itemize}
\item Set a power supply to the appropriate voltage to dissipate $\sim$2W in a 50$\Omega$ resistor (make
sure it is a beefy resistor rated to 2W, and do not exceed that threshold, or meltdown will ensue).
\item Using the Stefan-Boltzmann Law, calculate what temperature you think the resistor will hit.
\item Use an infrared thermometer to measure the temperature.  Were you close?
\end{itemize}

\subsection*{Measure the Noise Figure}
\begin{itemize}
\item Determine the power out of the receiver for the input resistor at room temperature.  Ideally, you'd do this
with a power meter or a spectrum analyzer, but today, we'll do the poor man's method of eyeballing one
standard deviation from the mean on an oscilloscope.  This isn't easy, and I apologize for that.  To keep your
sanity, you should estimate errorbars on this measurement, so you can see the effect of error on
your bottom line.
\item As in the previous step, set the resistor dissipating 2W, and measure the power out of the receiver.
IMPORTANT: Use a blocking cap to prevent a large DC voltage from entering the amplifier chain.  Otherwise, you
could damage your amplifiers.  When in doubt, ask.
\item Since you already know the temperature the resistor hits when dissipating 2W, you should be able to
solve for the gain and the receiver temperature (and if you want, you can calculate your Y factor).  This
would be a good time to use your error bars to determine your confidence interval on your measurement
of the gain and receiver temperature.  Make a plot and draw some lines!
\item Convert your receiver temperature to a noise figure.
\item How much is your error on the reciever temperature reduced if you have a prior on the gain (which
you should look up from the data sheet)?
\item How does your measured receiver temperature compare to the noise temperature you'd expect given
your chain of amplifiers? 
\end{itemize}

\section{Demonstrate the Central Limit Theorem}
\begin{itemize}
\item Write a program (in the central\_limit.py file that you created in the astro250 repo under your name)
that shows that, in the large-N limit, adding samples drawn from non-Gaussian random
distributions converges to a Gaussian distribution.
\item Also show that the standard deviation of the mean of $N$ Gaussian-random samples decreases as $\sqrt{N}$.
If you are trying to characterize samples of an unknown character to show that they are noise-like, this
is called an Allen variance test.
\end{itemize}

\section{Numerical Demonstration of the Radiometer Equation}
\begin{itemize}
\item Write a program (in the radiometer\_test.py file in the astro250 repo under your name) that simulates 
random noise corresponding to a noise temperature, and shows that the radiometer equation works.  In particular,
prove (numerically) that it is, in fact, $\sqrt{Bt}$ and not $\sqrt{2BT}$ in the denominator of the
radiometer equation.
\end{itemize}

\end{document}
